DESCRIÇÃO DO PROBLEMA
Uma identidade é uma representação digital de uma entidade ativa no sistema e que está associada a um conjunto de valores de atributos. O gerenciamento de identidade (IdM) procura atender à proliferação de identidades e quando se leva em consideração o controle de acesso temos os sistemas de Identity and Access Management (IAM). Provedores de serviço podem estabelecer uma relação de confiança com provedores de identidade para minimizar the spread of user credentials (username and password, for example). A privacidade pode ser obtida em diferentes graus através de acordos, protocolos e sistemas definidos entre as partes. Provedores podem querer estabeler políticas de controle de acesso baseadas em atributos e somente permitir que usuários acessem determinados recursos conforme os valores das informações pessoais armazenadas em provedores de identidade. O usuário pode querer obter acesso ao serviço mesmo não divulgando o valor do atributo requerido, mas o resultado da avaliação do mesmo. O contexto da proposta é o provedor de serviço adotando o padrão XACML para controle de acesso e um provedor de identidade executando o OpenID Connect.


extends OAuth to enable various use cases for resource owner-managed access 

In short, the UMA protocol lets you add interoperable authorization, access control, privacy, and consent features to your application ecosystem.

in 2013 and recommended that ONC support the developing and piloting of standards including OpenID Connect and OAuth2.0. In 2015, the HITSC recommended tracking development and piloting of new and emerging technology specifications including the User Managed Access (UMA) profile of OAuth2.0 for obtaining consumer consent. The HEART WG has developed a set of privacy and security specifications (HEART implementation specifications) using the following open standards: OAUTH 2.0, OpenID Connect and User Managed Access (UMA). These specifications enable an individual to control the authorization of access to health-related data sharing APIs. The goal of this Challenge is to incentive participants to create a Solution that utilizes the HEART implementation specifications to enable individuals to securely authorize the movement of their health data to destinations they choose. 


User-Managed Access (UMA) is a profile of OAuth 2.0. UMA defines how resource owners can control protected-resource access by clients operated by arbitrary requesting parties, where the resources reside on any number of resource servers, and where a centralized authorization server governs access based on resource owner policy


  the   permits the control of authorization manager delegated to the user.  delegated do user is th
  . UMA defines how resource owners can control protected-resource access by clients operated by arbitrary requesting parties, where the resources reside on any number of resource servers, and where a centralized authorization server governs access based on resource owner policy 
  esources,  residing  on  any  number  of  host  sites,  
  through  a  centralized  authorization  manager  that  makes  
  access decisions based on user in
  structions [8]. This gives 
  users  the  required  flexibility  in  sharing  their  data  and  
  supports  them  in  their  participation  in  interactions  and  
  collaboration   on   the   Web.   It   also   supports   potential   
  requesters with accessing a user’s data.
  
  How can UMA make requesting parties adhere to the user's wishes for privacy and data usage control?
  https://docs.kantarainitiative.org/uma/draft-uma-trust.html
  
  The UMA protocol [10] is a profile of OAuth. It was designed to give an individual a unified control point for1 authorizing who and what can get access to his or her online personal data (such as identity attributes), content (such as photos), and services (such as creating status updates), no matter where those resources live online. Further, UMA allows the individual to configure the control point to test the requesting side's suitability for authorization, including identity (such as “Do you control the email address bob@gmail.com?”) and promises (such as “Do you agree to these nondisclosure terms?”). This is known as claims-gathering [11] and it has a role to play in data usage control.



\section{Example Scenario}

In this section we will illustrate the problem and the proposed solution through a sample scenario. The example permits demonstrate 

Scenario The Scenario dimension defines the primary untrusted actor and potential attacker in a privacy-sensitive information exchange operation.

The Untrusted Server scenario describes the case of a service provider that aims to gain more information about the service
consumers than necessary. This usually relates to the identity of the service consumers, leading to techniques such as
anonymous communication which can be realized by relying on intermediate proxies between a sender and a receiver
responsible for masking the sender's identity (cf. (Chaum,1981)). The main issue with anonymous communication is

O XACML não provê facilidades para acomodar certificados de atributos \citep{dagdee2011extending},

The Web 2.0 and cloud allowed users access services and share his and her data between applications, organizations and individuals. To protect users credentials the authentication and authorization were externalized of applications and several protocols and frameworks show up. Examples of authentication protocols for Web are the SAML, WS-Federation, WS-Trust, OpenID and OpenID Connect. For authorization there are two well known protocols, the OAuth 2.0 (Open Authorization) and the UMA.


O trabalho que mais se aproxima da proposta do presente projeto é o que Ma apresentou em \cite{ma2015cloud}, pelos seguintes motivos: usa padrões atuais, o \textit{OpenID Connect} para solução de federação de identidade e XACML para controle de acesso de granularidade fina. Tendo em vista que a proposta foi desenhada para controlar o acesso de pacientes, médicos e pesquisados a dados de diagnósticos por imagens, foi incluída a necessidade de diretivas de consentimento para regular com melhor granularidade as condições para acesso aos dados de imagem. 

Bayseana:classificação

classificação+ações

use case

paragrafo
- 1a. frase introducao
- 6 frase
- cada frase deve ter uma palavra da anterior:coesão

results
was: trabalhos já realizados
had been: proposta e achados

A important aspect related to authorization is the fact that for a correct decision, the entity must have all the relevant attributes required by the policy.

A privacidade pode ser obtida quando se aplica a propriedade da confidencialidade especificamente sobre dados pessoais. Como será detalhado mais adiante, os dados pessoais podem conter classificações mais precisas, como aquela definida em \cite{heurix2015taxonomy}.



================

%	booktitle={Computer and Information Technology; Ubiquitous Computing and Communications; Dependable, Autonomic and Secure Computing; Pervasive Intelligence and Computing (CIT/IUCC/DASC/PICOM), 2015 IEEE International Conference on},
pages={168--175},


organization={IEEE}


================
O ABAC possui vantagens bem definidas quando se trata de usar e aplicar um sistema de controle de acesso em ambientes computacionais atuais. The concept of fine-grained authorization, that permite modelar políticas de acesso considerando determinados atributos, o que quer dizer que não só importa quem o usuário é, mas sobre o que, quando, onde, porque e como. The possibility to include attributes of the environment (context) of the request, such time and location. This enables the request be made without the identification of the requester and it permits that the use of model to Service-Oriented Architecture (SOA).

The complexity of systems take the creation of Policy Based Access Control (PBAC) in that the security policy is defined by rules according to which access control must be regulated \citep{samarati2001access}. The definition of a policy language and specifying a system to enforce it permit that the authorization can be externalized of the application. In this work we will use the XACML that permits express policies for authorization.


==================

To localize our proposal in the context of related works we presents the table \ref{tab-tabela-comparativa} with the points 

\begin{table}
	% \begin{table}[h]
	\tiny
	\centering
	\caption{Tabela comparativa entre as principais características referentes aos trabalhos relacionados.}
	\label{tab-tabela-comparativa}
	\begin{tabular}{|m{6em}|c|c|c|c|c|c|c|c|c|}
		\hline  & GerenIdent & \textit{AuthZ}(fina/grossa) & \textit{AuthZ}(modelo) & PadrõesMercado & Protótipo & ImplemVincProjeto & Complexidade & RESTful Web API\\ 
		\hline \cite{kolter2007privacy}     & Não        & Fina   & XACML           & Não            & Não(Access e-Gov) & Sim(Access e-Gov) & S/I \\ 
		\hline \cite{ardagna2008privacy}    & Sim(PRIME) & Grossa & Privada         & Parcial        & Sim               & Sim(PRIME)        & Alta(1) \\ 
		\hline \cite{ardagna2010exploiting} & Sim(PRIME) & Grossa & Privada         & Parcial        & Sim               & Sim(PRIME)        & Alta(1) \\ 
		\hline \cite{ardagna2010enabling}   & Não        & Fina   & XACML           & Sim(XACML,SAML)  & Não               & Sim(PrimeLife)    & S/I \\ 
		\hline \cite{kounga2010extending}   & Não        & Fina   & XACML           & Sim            & Não               & Sim(IGF)          & S/I \\ 
		\hline \cite{chadwick2012privacy}   & Não        & Fina   & XACML           & Parcial        & Sim               & Sim(PERMIS)       & Baixa(2) \\ 
		\hline \cite{ma2015cloud}           & Não        & Fina   & XACML           & Sim            & Sim               & Sim               & Baixa \\ 
		\hline \textit{Shibbolet}           & Sim        & Grossa & Privado         & Sim(SAML)      & Sistema em produção & \-     & Baixa/média \\
		\hline \textit{OpenID Connect}      & Sim        & Grossa & OAuth 2.0       & Sim(OAuth2.0)  & Sistema em produção & \-     & Baixa \\    
		\hline Proposta                     & Sim        & Fina   & XACML           & Sim            & Sim               & Baixa             & Sim(3) \\
		\hline 
	\end{tabular} 
	\\ Fonte: o autor.
	\\ Observações: S/I sem informações.
	\\ (1): complexidade conforme definido em \cite{nogueira2014aprimoramento}.
	\\ (2): complexidade de execução baixa conforme resultados no artigo \cite{chadwick2012privacy}.
	\\ (3): expectativa considerando os protocolos e sistemas usados pela proposta.
	% \end{table}
\end{table}

Detalhamento dos aspectos analisados nas propostas:
\begin{enumerate}
	\item A
	\item Gerenciamento de Identidade: verifica se a proposta inclui o relacionamento com um sistema de gerenciamento de identidade, ou seja, se há alguma sistema que fornece os atributos para autorização.
	\item Autorização (\textit{AuthZ}) (fina/grossa): especifica se a autorização é considerada sobre os atributos individuais do sujeito, recurso e ambiente, consistente com o modelo ABAC, o qual determina a autorização de granularidade fina. Caso contrário, diz-se que o sistema implementa autorização de granularidade grossa.
	\item Autorização (modelo): quando a proposta define um sistema de autorização, verifica se a mesma se baseia em algum modelo ou é proprietário à proposta.
	\item Padrões de mercado: algumas propostas podem fazer uso ou estender padrões ou especificações de mercado para mecanismos de controle de acesso, autorização e transporte de mensagens. As especificações mais adotadas nesse ramos de estudos são: XACML, SAML, OAuth, OpenID e \textit{OpenID Connect}.
	\item Protótipo: se foi ou não criado um protótipo com alguns resultados de validação.
	\item Implementação vinculada à projeto: alguns artigos apresentaram propostas dentro de projetos aos quais estavam vinculados. Um aspecto negativo desses artigos é que muitas suposições e contextos estão vinculados às definições dos projetos e/ou sistemas, dificultando ao pesquisador obter e testar as propostas
	\item Complexidade: essa característica está ligada tanto à implantação da proposta quanto à complexidade de execução.	
\end{enumerate}
