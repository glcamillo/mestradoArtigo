DESCRIÇÃO DO PROBLEMA
Uma identidade é uma representação digital de uma entidade ativa no sistema e que está associada a um conjunto de valores de atributos. O gerenciamento de identidade (IdM) procura atender à proliferação de identidades e quando se leva em consideração o controle de acesso temos os sistemas de Identity and Access Management (IAM). Provedores de serviço podem estabelecer uma relação de confiança com provedores de identidade para minimizar the spread of user credentials (username and password, for example). A privacidade pode ser obtida em diferentes graus através de acordos, protocolos e sistemas definidos entre as partes. Provedores podem querer estabeler políticas de controle de acesso baseadas em atributos e somente permitir que usuários acessem determinados recursos conforme os valores das informações pessoais armazenadas em provedores de identidade. O usuário pode querer obter acesso ao serviço mesmo não divulgando o valor do atributo requerido, mas o resultado da avaliação do mesmo. O contexto da proposta é o provedor de serviço adotando o padrão XACML para controle de acesso e um provedor de identidade executando o OpenID Connect.


extends OAuth to enable various use cases for resource owner-managed access 

In short, the UMA protocol lets you add interoperable authorization, access control, privacy, and consent features to your application ecosystem.

in 2013 and recommended that ONC support the developing and piloting of standards including OpenID Connect and OAuth2.0. In 2015, the HITSC recommended tracking development and piloting of new and emerging technology specifications including the User Managed Access (UMA) profile of OAuth2.0 for obtaining consumer consent. The HEART WG has developed a set of privacy and security specifications (HEART implementation specifications) using the following open standards: OAUTH 2.0, OpenID Connect and User Managed Access (UMA). These specifications enable an individual to control the authorization of access to health-related data sharing APIs. The goal of this Challenge is to incentive participants to create a Solution that utilizes the HEART implementation specifications to enable individuals to securely authorize the movement of their health data to destinations they choose. 